As soon as two entities need to interact, there is need for a protocol which 
secures the communication.
This can be inside a system or even one entity communicating with itself in 
different points in time, which is the case when storing something for use at 
a later time.
Protocols can be easy to verify manually if they are simple, but even then, as 
we will see, it is easy to miss something.
We will discuss how to design secure protocols and introduce some tools for 
verification, e.g.\ automated formal verification.

More concretely, after this session you should be able to
\begin{itemize}
  \item \emph{understand} the different approaches and their limits to verify 
    the security of protocols.
\end{itemize}

Anderson gives an overview of this area in 
\citetitle{Anderson2008sea}~\cite{Anderson2008sea}, Chapter 
3 \enquote{Protocols}.
Gollmann has a more technically oriented treatment of a part of this topic in 
Chapter 15 of \citetitle{Gollmann2011cs}~\cite{Gollmann2011cs}.
To complement these texts we will also touch upon some of the material in 
\citetitle{ProVerif}~\cite{ProVerif}.

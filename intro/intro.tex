% $Id$
%\documentclass[handout]{beamer}
\documentclass{beamer}
\usepackage[utf8]{inputenc}
\usepackage[T1]{fontenc}
\usepackage[english,swedish]{babel}
\usepackage{url}
\usepackage{graphicx}
\usepackage{color}
\usepackage{subfig}
\usepackage{multicol}
\usepackage{amssymb,amsmath,amsthm}
\usepackage{stmaryrd}
\usepackage{booktabs}
\usepackage[binary,squaren]{SIunits}
\usepackage{csquotes}

\usepackage{listings}
\lstset{basicstyle=\small,breaklines=true}

\usepackage{xparse}
\ProvideDocumentEnvironment{exercise}{o}{%
  \setbeamercolor{block body}{bg=yellow!30,fg=black}
  \setbeamercolor{block title}{bg=yellow,fg=black}
  \IfValueTF{#1}{%
    \begin{block}{Exercise: #1}
  }{%
    \begin{block}{Exercise}
  }
}{%
  \end{block}
}

\mode<presentation>{%
  \usetheme{Berlin}
  \setbeamertemplate{footline}{%
    \begin{beamercolorbox}[colsep=1.5pt]{upper separation line foot}
    \end{beamercolorbox}
    \begin{beamercolorbox}[ht=2.5ex,dp=1.125ex,%
      leftskip=.3cm,rightskip=.3cm plus1fil]{author in head/foot}%
      \leavevmode{\usebeamerfont{author in head/foot}\insertshortauthor}%
      \hfill%
      {\usebeamerfont{institute in head/foot}%
        \usebeamercolor[fg]{institute in head/foot}\insertshortinstitute}%
    \end{beamercolorbox}%
    \begin{beamercolorbox}[ht=2.5ex,dp=1.125ex,%
      leftskip=.3cm,rightskip=.3cm plus1fil]{title in head/foot}%
      {\usebeamerfont{title in head/foot}\insertshorttitle}%
      \hfill\insertframenumber%
    \end{beamercolorbox}%
    \begin{beamercolorbox}[colsep=1.5pt]{lower separation line foot}
    \end{beamercolorbox}
  }
  \setbeamercovered{transparent}
  \setbeamertemplate{bibliography item}[text]
}\setbeamertemplate{bibliography item}[text]

\usepackage[natbib,style=alphabetic,maxbibnames=99]{biblatex}
\addbibresource{intro.bib}

%\newtheorem{theorem}{Sats}%[chapter]
\providetranslation[to=swedish]{Theorem}{Sats}
%\newtheorem{lemma}{Lemma}
%\newtheorem{corollary}{Korollarium}
\providetranslation[to=swedish]{Corollary}{Korollarium}
\theoremstyle{definition}
%\newtheorem{definition}{Definition}
%\newtheorem{axiom}[definition]{Axiom}
\newenvironment{axiom}[1]{\begin{block}{Postulat (#1)}}{\end{block}}
%\newtheorem{example}{Exempel}
\providetranslation[to=swedish]{Example}{Exempel}
%\newtheorem{exercise}{Övning}
\theoremstyle{remark}
%\newtheorem{remark}{Anmärkning}

%\numberwithin{section}{chapter}
%\numberwithin{equation}{chapter}

\DeclareMathOperator{\p}{\mathcal{P}}
\let\P\p
\DeclareMathOperator{\C}{\mathcal{C}}
\DeclareMathOperator{\K}{\mathcal{K}}
\DeclareMathOperator{\E}{\mathcal{E}}
\DeclareMathOperator{\D}{\mathcal{D}}

\DeclareMathOperator{\N}{\mathbb{N}}
\DeclareMathOperator{\Z}{\mathbb{Z}}

\renewcommand{\p}{{P}}
\renewcommand{\c}{{C}}
\renewcommand{\k}{{K}}
\newcommand{\x}{{X}}
\newcommand{\y}{{Y}}
\newcommand{\e}{{E}}
\newcommand{\s}{{S}}

\DeclareMathOperator{\lequiv}{\Longleftrightarrow}
\DeclareMathOperator{\xor}{\oplus}

\renewcommand{\qedsymbol}{Q.E.D.}

% BAN logic
\DeclareMathOperator{\believes}{|\!\!\!\equiv}
\DeclareMathOperator{\said}{|\!\!\!\sim}
\DeclareMathOperator{\controls}{\Mapsto}
\DeclareMathOperator{\sees}{\lhd}
\newcommand{\fresh}[1]{\#(#1)}
\newcommand{\encrypt}[2]{\{#1\}_{#2}}
\newcommand{\share}[1]{\stackrel{#1}{\leftrightarrow}}
\newcommand{\pubkey}[1]{\stackrel{#1}{\mapsto}}

\title[Säkra protokoll]{%
  Säkra protokoll och procedurer
}
\author[D.~Bosk]{Daniel Bosk\footnote{%
  Detta verk är tillgängliggjort under licensen Creative Commons 
  Erkännande-DelaLika 2.5 Sverige (CC BY-SA 2.5 SE).
	För att se en sammanfattning och kopia av licenstexten besök URL 
	\url{http://creativecommons.org/licenses/by-sa/2.5/se/}.
}}
\institute[MIUN IKS]{%
  %Department of Information and Communication Systems (ICS),\\
  %Mid Sweden University, Sundsvall.
	%
  Avdelningen för informations- och kommunikationssytem (IKS),\\
  Mittuniversitetet, Sundsvall.
}
\date{\today}

%\pgfdeclareimage[height=0.65cm]{university-logo}{MU_logotyp_int_CMYK.pdf}
%\logo{\pgfuseimage{university-logo}}

\AtBeginSection[]{%
  \begin{frame}<beamer>{Översikt}
    \tableofcontents[currentsection]
  \end{frame}
}

\begin{document}

\begin{frame}
  \titlepage{}
\end{frame}


% Since this a solution template for a generic talk, very little can
% be said about how it should be structured. However, the talk length
% of between 15min and 45min and the theme suggest that you stick to
% the following rules:  

% - Exactly two or three sections (other than the summary).
% - At *most* three subsections per section.
% - Talk about 30s to 2min per frame. So there should be between about
%   15 and 30 frames, all told.


\section{Introduktion}

\subsection{Vad är ett protokoll?}

\begin{frame}
  \begin{itemize}
    \item Ett system består av en uppsättning principals.
    \item Ett protokoll är en uppsättning regler som styr hur dessa 
      kommunicerar.
  \end{itemize}
  \begin{example}[Tentamen MIUN]
    \begin{enumerate}
      \item Tentamensvakten öppnar salen och ger varje tentand ett nummer.
      \item Tentanden går in och sätter sig vid sin tilldelade plats.
      \item Efter att tentan börjat jämför tentamensvakten tentandens 
        legitimation och nummer.
      \item Vid inlämning av skrivning jämförs legitimationen och numret.
    \end{enumerate}
  \end{example}
\end{frame}

\begin{frame}
  \begin{itemize}
    \item Bör vara designade för att motstå attacker.
    \item Både oavsiktligt och avsiktligt brott mot protokollet.
  \end{itemize}
\end{frame}

\begin{frame}
  \begin{example}{Beställa vin}
    \begin{enumerate}
      \item Hovmästaren visar vinlistan för värden.
      \item Värden väljer vin, hovmästaren hämtar.
      \item Värden provsmakar vin, det serveras till gästerna.
    \end{enumerate}
  \end{example}
  \begin{block}{Egenskaper}
    \begin{description}
      \item[Konfidentialitet] Gästerna får ej veta priset.
      \item[Riktighet] Hovmästaren kan inte byta ut vinet.
      \item[Oavvislighet] Värden kan inte falskt klaga på vinet i efterhand.
    \end{description}
  \end{block}
\end{frame}

\begin{frame}
  \begin{example}[Tentamen MIUN]
    \begin{enumerate}
      \item Tentamensvakten öppnar salen och ger varje tentand ett nummer.
      \item Tentanden går in och sätter sig vid sin tilldelade plats.
      \item Efter att tentan börjat jämför tentamensvakten tentandens 
        legitimation och nummer.
      \item Vid inlämning av skrivning jämförs legitimationen och numret.
    \end{enumerate}
  \end{example}
  \begin{block}{Egenskaper}
    \begin{description}
      \item[Anonymitet] Varje tentand förblir anonym.
      \item[Autenticitet] Skrivningen är garanterat associerad med tentanden.
    \end{description}
  \end{block}
\end{frame}

\begin{frame}
  \begin{itemize}
    \item Konstrueras utifrån grundläggande antaganden.
      \begin{itemize}
        \item Exempelvis att kortägaren kan mata in PIN-koden direkt 
          i terminalen.
    \end{itemize}
    \item Analysera om hoten är rimliga.
    \item Analysera om protokollet hanterar dem.
  \end{itemize}
\end{frame}

\subsection{Ibland blir det fel}

\begin{frame}
  \begin{example}[Tentamen KTH]
    KTH:s tentamensvakter hade inte sista punkten i tentamensprotokollet.
    \begin{enumerate}
      \item Tentamensvakten öppnar salen.
      \item Tentanden går in och sätter sig.
      \item Efter att tentan börjat undersöker tentamensvakten tentandens 
        legitimation och antecknar plats i salen.
      \item Tentanden lämnar in skrivningen.
    \end{enumerate}
  \end{example}
  \begin{block}{Egenskaper}
    Autentiseringen brister, tentanden kan skriva vilket namn och personnumer 
    som helst på inlämnad tentamen.
    Tentanden är inte garanterad anonymitet om den inte explicit ber om att 
    tentan ska skrivas anonymt.
  \end{block}
\end{frame}

\begin{frame}
  \begin{block}{Autentisera uttag}
    \begin{itemize}
      \item Banker lagrade kontonummer på magnetremsan.
      \item PIN-koden skickades till centrala systemet för verifiering.
    \end{itemize}
  \end{block}
  \begin{block}{''Förbättring''}
    Kryptera PIN-koden och lagra den på magnetremsan så att uttagsautomaten kan 
    verifiera den när den inte får kontakt med centrala systemet.
  \end{block}
\end{frame}

\begin{frame}
  \begin{block}{Problem}
    \begin{itemize}
      \item Jag kan byta ut kontonumret men inte ändra koden.
      \item Ändra kontonummer och använd min egen kod för att ta ut från annans 
        konto.
    \end{itemize}
  \end{block}
\end{frame}

\subsection{Autentisering}

\begin{frame}
  \begin{itemize}
    \item Lösenord och PIN-koder är fundamentala metoder för autentisiering.

    \item Exempelvis 90-talets fjärrlås till bilen:
      \begin{itemize}
        \item Nyckeln skickade serienumret.
        \item Bilen kontrollerade allt den mottog och jämförde med sitt 
          serienummer.
      \end{itemize}

    \item Kunde angripas med en inspelningsattack.
      \begin{itemize}
        \item Spela in allt som sänds (serienummer).
        \item Spela upp serienummer för att låsa upp.
      \end{itemize}

  \end{itemize}
\end{frame}

\begin{frame}
  \begin{itemize}
    \item Ibland kan ett enkelt lösenord vara rätt väg att gå.

    \item Matkuponger på lunchställen:
      \begin{itemize}
        \item Papperslapp med serienummer.
        \item Kan framställas i kopiatorn.
      \end{itemize}

    \item Hotet är inte värt högre säkerhet.

  \end{itemize}
\end{frame}


\section{Formell notation}

\subsection{Protokoll}

\begin{frame}
  \begin{example}[Protokollbeskrivning]
    Två principals \(P, P^\prime\) ska kommunicera.
    \begin{enumerate}
      \item \(P\) skickar sitt namn till \(P^\prime\).
      \item \(P^\prime\) svarar med ett token \(t_P\) för vidare användning, 
        detta är krypterat med \(P\):s kryptonyckel \(k_P\).
    \end{enumerate}
  \end{example}
  \begin{example}[Formell beskrivning]
    Principals \(P, P^\prime\), token \(t_P\), \(P\):s kryptonyckel \(k_P\).
    \begin{align*}
      P\to P^\prime&\colon P \\
      P^\prime\to P&\colon \{t_P\}_{k_P}
    \end{align*}
  \end{example}
\end{frame}

\begin{frame}{Tentamen}
  \begin{example}[Autentisering MIUN]
    Låt \(T\) vara tentanden, \(V\) tentamensvakten, \(n_T\) det unika numret 
    för \(T\) och \(S\) skrivningen.
    Vidare låt \(k\) vara en kryptonyckel delad mellan legitimationsutfärdaren 
    och tentamensvakten (legitimation).
    \begin{align*}
      V\to T&\colon n_T \\
      T\to V&\colon \{T\}_k, n_T \\
      T\to V&\colon \{T\}_k, n_T, S
    \end{align*}
  \end{example}
\end{frame}

\begin{frame}{Tentamen}
  \begin{example}[Autentisering KTH]
    Låt \(T\) vara tentanden, \(V\) tentamensvakten och \(S\) skrivningen.
    Vidare låt \(k\) vara en kryptonyckel delad mellan legitimationsutfärdaren 
    och tentamensvakten (legitimation).
    \begin{align*}
      T\to V&\colon T, \{T\}_k \\
      T\to V&\colon T, S
    \end{align*}
  \end{example}
\end{frame}

\begin{frame}
  \begin{example}[Needham-Schröder Shared-Key]
    Låt \(A, B, S\) vara principals, \(S\) en server, \(n_A, n_B\) vara 
    nonces\footnote{Number used once.}.
    \begin{align*}
      A\to S &\colon A, B, n_A \\
      S\to A &\colon \encrypt{n_A, B, k_{AB}, \encrypt{k_{AB}, 
      A}{k_{BS}}}{k_{AS}} \\
      A\to B &\colon \encrypt{k_{AB}, A}{k_{BS}} \\
      B\to A &\colon \encrypt{n_B}{k_{AB}} \\
      A\to B &\colon \encrypt{n_B-1}{k_{AB}}
    \end{align*}
  \end{example}
\end{frame}

\subsection{BAN-logik}

\begin{frame}
  \begin{itemize}
    \item Namn efter upphovsmän: Burrows, Abadi och Needham.
    \item Presenterades i en artikel~\cite{BAN90alo} från 1989.
    \item Används för formell analys av autentiseringsprotokoll.
    \item Har väsentligen utökats~\cite{Syverson2001tlo}.
  \end{itemize}
\end{frame}

\begin{frame}{Definitioner}
  \begin{description}
    \item[\(A\believes X\)] \(A\) tror på \(X\), och kan agera som att \(X\) 
      vore sann.
    \item[\(A\said X\)] \(A\) sade \(X\), eller något som innehöll \(X\), vid 
      något tidigare tillfälle.
      \(A\) trodde då på \(X\) och förstod att \(X\) skickades.
    \item[\(A\sees X\)] \(A\) ser \(X\), eller har mottagit ett meddelande 
      innehållandes \(X\).
      Detta kan innebära att avkryptera.
    \item[\(A\controls X\)] \(A\) är auktoritet över \(X\).
      \(A\) går att lita på gällandes \(X\).
    \item[\(\fresh X\)] \(X\) är färsk, det vill säga inte en återuppspelning 
      från tidigare protokollsession.
  \end{description}
\end{frame}

\begin{frame}{Definitioner}
  \begin{description}
    \item[\(A\share{k} B\)] \(A\) och \(B\) delar nyckeln \(k\).
      \(k\) är giltig för kommunikation mellan \(A\) och \(B\).
    \item[\(\pubkey{k} A\)] \(A\) har den publika nyckeln \(k\).
      Den motsvarande privata nyckeln \(k^{-1}\) hålls hemlig.
    \item[\(\encrypt{X}{k}\)] \(X\) är krypterad med nyckeln \(k\).
      Principals känner igen egna meddelanden och krypterade meddelanden kan då 
      användas för att säkert identifiera avsändaren.
  \end{description}
\end{frame}

\begin{frame}{Postulat}
  Har följande sex postulat:
  \begin{itemize}
    \item Message meaning,
    \item Nonce verification,
    \item Jurisdiction,
    \item Belief conjuncatenation,
    \item Freshness conjuncatenation,
    \item Seeing is receiving.
  \end{itemize}
\end{frame}

\begin{frame}{Postulat}
  \begin{axiom}{Message meaning}
    Låt \(A\) och \(B\) vara principals, \(k\) är en delad nyckel och \(X\) är 
    ett uttalande.
    Då gäller
    \begin{align*}
      \frac{A\believes A\share{k} B, A\sees \encrypt{X}{k}}%
        {A\believes B\said X}.
    \end{align*}
    Om \(A\) tror att \(k\) är giltig nyckel för att kommunicera med \(B\) och 
    \(A\) har mottagit \(X\) krypterat med \(k\),
    då kan \(A\) tro att \(B\) har sagt \(X\).
  \end{axiom}
\end{frame}

\begin{frame}{Postulat}
  \begin{axiom}{Nonce verification}
    Låt \(A\) och \(B\) vara principals, \(X\) är ett uttalande.
    Då gäller
    \begin{align*}
      \frac{A\believes \fresh X, A\believes B\said X}%
        {A\believes B\believes X}.
    \end{align*}
    Om \(A\) tror att \(X\) är färsk och \(A\) tror att \(B\) har sagt \(X\),
    då kan \(A\) tro att \(B\) tror på \(X\).
  \end{axiom}
\end{frame}

\begin{frame}{Postulat}
  \begin{axiom}{Jurisdiction}
    Låt \(A\) och \(B\) vara principals, \(K\) är en delad nyckel och \(X\) är 
    ett uttalande.
    Då gäller
    \begin{align*}
      \frac{A\believes B\controls X, A\believes B\believes X}%
        {A\believes X}.
    \end{align*}
    Om \(A\) tror att \(B\) kontrollerar \(X\) och \(A\) tror att \(B\) tror på 
    \(X\),
    då kan \(A\) tro på \(X\).
  \end{axiom}
\end{frame}

\begin{frame}{Analys}
  \begin{itemize}
    \item Måste idealisera protkollet, skriva det i termer av BAN\@.
    \item Måste identifiera antaganden.
    \item Annotera protokollet.
    \item Använd logiken för att härleda vad principals tror på.
  \end{itemize}
  För detaljer och exempel, se~\cite{Anderson2008sea} avsnitt 3.8 
  och~\cite{Syverson2001tlo} avsnitt 2.3.
\end{frame}

% XXX prove exam protocol
%\begin{frame}{Exempel}
%  \begin{block}{Återigen tentamen vid MIUN}
%    Låt \(T\) vara tentanden, \(V\) tentamensvakten, \(n_T\) det unika numret 
%    för \(T\) och \(S\) skrivningen.
%    Vidare låt \(k\) vara en kryptonyckel delad mellan legitimationsutfärdaren 
%    och tentamensvakten (legitimation).
%    \begin{align*}
%      V\to T&\colon n_T \\
%      T\to V&\colon \{T\}_k, n_T \\
%      T\to V&\colon \{T\}_k, n_T, S
%    \end{align*}
%  \end{block}
%\end{frame}
%
%\begin{frame}{Exempel}
%  \begin{block}{Idealiserad form}
%    Tentamensförfarandet på idealiserad form:
%    \begin{align*}
%      T\to V&\colon \{T\}_k, \fresh n_T \\
%      T\to V&\colon \{T\}_k, \fresh n_T, S
%    \end{align*}
%  \end{block}
%  \begin{block}{Antaganden}
%    Vi gör följande antaganden:
%    \begin{enumerate}
%      \item \(V\believes V\share{k} \text{Skatteverket}\).
%      \item \(V\believes T\controls S\).
%      \item \(V\believes \fresh n_T\).
%    \end{enumerate}
%  \end{block}
%\end{frame}
%
%\begin{frame}{Exempel}
%  \begin{block}{Vill visa}
%    \(V\believes (n_T, S)\).
%  \end{block}
%  \begin{proof}
%  \end{proof}
%\end{frame}

% XXX prove Needham-Schröder
%\begin{frame}{Exempel}
%\end{frame}

\begin{frame}
  \begin{itemize}
    \item Med BAN-logik kunde man visa att Needham-Schröder-protokollet krävde 
      antagandet \(B\believes \fresh{A\share{k_{AB}} B}\).
    \item Ledde till Denning-Sacco-attacken.
  \end{itemize}
  \begin{block}{Denning-Sacco}
    Låt \(E_A\) beteckna angriparen som låtsas vara \(A\).
    \begin{align*}
      E_A\to B &\colon \encrypt{k_{AB}, A}{k_{BS}} \\
      B\to E_A &\colon \encrypt{n_B^\prime}{k_{AB}} \\
      E_A\to B &\colon \encrypt{n_B^\prime-1}{k_{AB}}
    \end{align*}
  \end{block}
\end{frame}

\begin{frame}{Begränsningar}
  \begin{itemize}
    \item Våra externa antaganden är ett problem: antag att nyckeln inte är 
      tillgänglig för obehöriga.

    \item Kan bli problem vid idealiseringen av protokollet.

  \end{itemize}
\end{frame}

\subsection{Automated Verification}

\begin{frame}
  \begin{block}{Automated Theorem Proving}
    \begin{itemize}
      \item Russel and Whitehead's Principia Mathematica from 1910.
      \item Löwenheim-Skolem theorem from 1920.
      \item Herbrand universe and interpretation from 1930.
      \item Gödel's On Formally Undecidable Propositions of Principia 
        Mathematica and Related Systems from 1931.
      \item Hence automated theorem proving works for some theorems, but not 
        all, it depends on the decidability of the problem.
    \end{itemize}
  \end{block}
\end{frame}

\begin{frame}
  \begin{block}{Proof Verification}
    \begin{itemize}
      \item This is a simpler problem where we verify a proof.
      \item We do this by having each step of the proof verifiable by 
        a function.
    \end{itemize}
  \end{block}
\end{frame}

\begin{frame}
  \begin{block}{Automated Tools}
    \begin{itemize}
      \item There exists numerous tools, e.g.\ ProVerif~\cite{ProVerif}.
      \item These are generally based on an equational theory.
      \item Then the equations are used to verify that the security properties 
        hold.
      \item We generate a proof that our properties cannot be broken under the 
        assumptions of the equational theory.
    \end{itemize}
  \end{block}
\end{frame}

\begin{frame}[fragile]
  \begin{example}[Public-key crypto with ProVerif]
    \begin{lstlisting}
type skey.
type pkey.
type seed.
type block.
type encblock.

(* Probabilistic public-key encryption *)

fun pk(skey): pkey.
fun enc(block, pkey, seed): encblock.
fun dec(encblock, skey): block.
equation forall x: block, y: skey, z: seed;  dec(enc(x, pk(y), z), y) = x.
    \end{lstlisting}
  \end{example}
\end{frame}

\begin{frame}
  \begin{itemize}
    \item The idea is similar to that of BAN-logic.
    \item However, usually these automated verification tools verify that we 
      are not vulnerable to known attacks.
  \end{itemize}
\end{frame}


\section{Protokoll och attacker}

\subsection{Enkel autentisering}

\begin{frame}{En bättre metod för fjärrlås}
  \begin{example}[Fjärrlås]
    Låt \(A, B\) vara principals, \(n\) nonce, \(k_A\) en nyckel unik för 
    \(A\).
    \begin{align*}
      A\to B\colon A, \encrypt{A, n}{k_A}
    \end{align*}
  \end{example}
  \begin{block}{Egenskaper}
    \begin{itemize}
      \item Nonce \(n\) för färskhet.
      \item Krypteringen för identifiering.
    \end{itemize}
  \end{block}
\end{frame}

\begin{frame}{Nyckelhantering}
  \begin{itemize}
    \item Måste hantera nycklarna \(k_i\) för alla enheter \(i\).
    \item \emph{Nyckeldiversifiering}: huvudnyckel \(k_M\) och generera \(k_i 
      = \encrypt{i}{k_M}\).
    \item Måste tänka efter:
      \begin{itemize}
        \item 128-bitar nyckel krypterar 16-bitar ID, mindre lämpligt för 
          diversifiering.
        \item Svagt chiffer ger också dåligt resultat.
        \item \(k_i = i\xor k_M\)?
      \end{itemize}
  \end{itemize}
\end{frame}

\begin{frame}{Kolla nonces}
  Kolla nonces långt tillbaka i tiden.
  \begin{itemize}
    \item Jämför med senaste nonce.
    \item Spela in två och spela upp dem varannan gång.
    \item Förbetalda elmätare, köp två laddningar och använd dem om vartannat.
  \end{itemize}
\end{frame}

\begin{frame}{Betjäntattacken}
  \begin{itemize}
    \item Hur genereras nonces?
    \item En person som har tillfällig åtkomst att generera tokens.
    \item Generera ett antal, använd dem senare.
    \item Exempelvis engångskoder för att logga in hos internetbanken.
    \item Attacken fungerar om nonces är (pseudo-) slumptal.
  \end{itemize}
\end{frame}

\begin{frame}{Kontra betjäntattacken}
  \begin{block}{Förbättring}
    \begin{itemize}
      \item Använd en räknare \(c\) som successivt ökas på.
      \item \(A\to B\colon A, \encrypt{A, c+1}{k_A}\), \(c = c+1\).
      \item Inget \(c^\prime \leq c\) accepteras.
    \end{itemize}
  \end{block}
  \begin{block}{Problem}
    \begin{itemize}
      \item Får inte ha jämförelsen \(c^\prime = c\), ger 
        synkroniseringsproblem.
      \item \(c\notin \Z_+\) utan \(c\in \Z_{2^x}\), för något \(x\in \N\): vid 
        något tillfälle blir då \(c+1 < c \pmod{2^x}\).
    \end{itemize}
  \end{block}
\end{frame}

\begin{frame}{Andra tillämpningar}
  \begin{itemize}
    \item Tillbehörskontroll: skrivare ändrar inställning från \unit{1200}{dpi} 
      till \unit{300}{dpi} om icke-originalbläckpatroner används.
    \item \enquote{Använd alltid godkända originaldelar}.
    \item Inte hålla angripare ute, utan hålla användare inne.
    \item Läs kapitel 7 \emph{Economics} i~\cite{Anderson2008sea} för vidare 
      diskussion.
  \end{itemize}
\end{frame}

\subsection{Challenge--response}

\begin{frame}
  \begin{block}{Grundläggande princip}
    Två principals \(A, B\) med gemensam nyckel \(k\) och nonce \(n\).
    \begin{align*}
      A\to B &\colon n \\
      B\to A &\colon \encrypt{B, n}{k}
    \end{align*}
  \end{block}
  \begin{block}{Problem}
    \begin{itemize}
      \item Dåliga (pseudo-) slumptalsgeneratorer, ger förutsägbara \(n\).
    \end{itemize}
  \end{block}
\end{frame}

\begin{frame}{Tvåfaktorautentisering}
  \begin{itemize}
    \item Ha användarnamn och lösenord.
    \item Komplettera med extern kod; exempelvis genererad av koddosa, SMS till 
      mobiltelefonen.
    \item Finns många varianter, kombinera två:
      \begin{itemize}
        \item Något du vet (lösenord),
        \item något du har (koddosa, mobiltelefon),
        \item något du är (biometrik).
      \end{itemize}
  \end{itemize}
\end{frame}

\begin{frame}{Tvåfaktorautentisering}
  \begin{block}{Protokoll (tvåfaktorautentisering med koddosa)}
    Låt \(A, B, D\) vara principals, \(D\) är koddosa, \(k\) är nyckel delad 
    mellan \(B, D\) och \(p\) är \(A\):s PIN-kod.
    \begin{align*}
      A\to B &\colon A \\
      B\to A &\colon n \\
      A\to D &\colon n, p \\
      D\to A &\colon \encrypt{n}{k} \\
      A\to B &\colon \encrypt{n}{k}
    \end{align*}
  \end{block}
\end{frame}

\begin{frame}{Tvåkanalsautentisering}
  \begin{block}{Protokoll (tvåkanalsautentisering med mobiltelefon)}
    Låt \(A, B, M\) vara principals, \(M\) är mobiltelefon och \(p\) är \(A\):s 
    lösenord.
    \begin{align*}
      A\to B &\colon A, p \\
      B\to M &\colon n \\
      M\to A &\colon n \\
      A\to B &\colon n
    \end{align*}
  \end{block}
\end{frame}

\subsection{Miljöbyte}

\begin{frame}
  \begin{itemize}
    \item Betalkortsystemet designades för en pålitlig miljö.
    \item Kraftigt reglerad miljö inbyggd i bankens fasad.
    \item Tillämpas i den mindre pålitliga miljön i samtliga affärer.
    \item Skimming.
  \end{itemize}
\end{frame}

\begin{frame}{Personen i mitten}
  \begin{itemize}
    \item \enquote{Det är enkelt att spela oavgjort mot en schackstormästare 
        i postschack: spela bara mot två stormästare samtidigt, en som vit och 
        en som svart, och skicka deras brev mellan varandra.} (John Convey)
    \item Problem med pålitliga användargränssnitt: hur vet du att inte 
      kortterminalen ljuger?
  \end{itemize}
\end{frame}

\subsection{Internetbanken och betalkort}

\begin{frame}{Olika former av bankdosor}
  \begin{block}{Swedbank}
    \begin{itemize}
      \item Individuell dosa, förkonfigurerad av banken.
      \item Kan generera engångskod.
      \item Kan hantera challenge--response.
    \end{itemize}
  \end{block}
  \begin{block}{Nordea}
    \begin{itemize}
      \item Oberoende smartkortläsare, använder individuellt betalkort.
      \item Kan generera engångskod.
      \item Kan hantera challenge--response.
    \end{itemize}
  \end{block}
\end{frame}

\begin{frame}{Problem som kan uppstå}
  \begin{block}{Problem}
    \begin{itemize}
      \item Om bankkort och dosa förvaras tillsammans kan PIN-koden utläsas 
        från de slitna knapparna på bankdosan.
      \item Om kortet används i en dålig terminal har angriparna allt som 
        behövs för att logga in till ditt bankkonto.
    \end{itemize}
  \end{block}
  \begin{block}{Förbättringar}
    \begin{itemize}
      \item Använd inte samma säkerhetsmekanism i flera sammanhang.
      \item Ha separata oberoende mekanismer.
      \item Ha ett pålitligt användargränssnitt.
    \end{itemize}
  \end{block}
\end{frame}

% XXX add slides about BankID
%\subsection{BankID}
%\begin{frame}
%\end{frame}
%\begin{frame}{Att lämna in deklarationen}
%\end{frame}


%%%%%%%%%%%%%%%%%%%%%%

\begin{frame}{Referenser}
	\small
  \printbibliography{}
\end{frame}

\end{document}


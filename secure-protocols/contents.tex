\mode*

% Since this a solution template for a generic talk, very little can
% be said about how it should be structured. However, the talk length
% of between 15min and 45min and the theme suggest that you stick to
% the following rules:  

% - Exactly two or three sections (other than the summary).
% - At *most* three subsections per section.
% - Talk about 30s to 2min per frame. So there should be between about
%   15 and 30 frames, all told.


\section{Secure protocols}

\subsection{Vad är ett protokoll?}

\begin{frame}
  \begin{definition}[Protokoll]
    \begin{itemize}
      \item Ett system består av en uppsättning principals.
      \item Ett protokoll är en uppsättning regler som styr hur dessa 
        kommunicerar.
    \end{itemize}
  \end{definition}

  \pause{}

  \begin{example}[Tentamen MIUN]
    \begin{enumerate}
      \item Tentamensvakten öppnar salen och ger varje tentand ett nummer.
      \item Tentanden går in och sätter sig vid sin tilldelade plats.
      \item Efter att tentan börjat jämför tentamensvakten tentandens 
        legitimation och nummer.
      \item Vid inlämning av skrivning jämförs legitimationen och numret.
    \end{enumerate}
  \end{example}
\end{frame}

\begin{frame}
  \begin{alertblock}{Att tänka på}
    \begin{itemize}
      \item Bör vara designade för att motstå attacker.
      \item Både oavsiktligt och avsiktligt brott mot protokollet.
    \end{itemize}
  \end{alertblock}
\end{frame}

\begin{frame}
  \begin{example}[Beställa vin]
    \begin{enumerate}
      \item Hovmästaren visar vinlistan för värden.
      \item Värden väljer vin, hovmästaren hämtar.
      \item Värden provsmakar vin, det serveras till gästerna.
    \end{enumerate}
  \end{example}

  \pause{}

  \begin{block}{Egenskaper}
    \begin{description}
      \item[Konfidentialitet] Gästerna får ej veta priset.
      \item[Riktighet] Hovmästaren kan inte byta ut vinet.
      \item[Oavvislighet] Värden kan inte falskt klaga på vinet i efterhand.
    \end{description}
  \end{block}
\end{frame}

\begin{frame}
  \begin{example}[Tentamen MIUN]
    \begin{enumerate}
      \item Tentamensvakten öppnar salen och ger varje tentand ett nummer.
      \item Tentanden går in och sätter sig vid sin tilldelade plats.
      \item Efter att tentan börjat jämför tentamensvakten tentandens 
        legitimation och nummer.
      \item Vid inlämning av skrivning jämförs legitimationen och numret.
    \end{enumerate}
  \end{example}

  \pause{}

  \begin{block}{Egenskaper}
    \begin{description}
      \item[Anonymitet] Varje tentand förblir anonym.
      \item[Autenticitet] Skrivningen är garanterat associerad med tentanden.
    \end{description}
  \end{block}
\end{frame}

\begin{frame}
  \begin{alertblock}{Att tänka på}
    \begin{itemize}
      \item Konstrueras utifrån grundläggande antaganden.
        \begin{itemize}
          \item Exempelvis att kortägaren kan mata in PIN-koden direkt 
            i terminalen.
        \end{itemize}
      \item Analysera om hoten är rimliga.
      \item Analysera om protokollet hanterar dem.
    \end{itemize}
  \end{alertblock}
\end{frame}

\subsection{Ibland blir det fel}

\begin{frame}
  \begin{example}[Tentamen KTH]
    \begin{enumerate}
      \item Tentamensvakten öppnar salen.
      \item Tentanden går in och sätter sig.
      \item Efter att tentan börjat undersöker tentamensvakten tentandens 
        legitimation och antecknar plats i salen.
      \item Tentanden lämnar in skrivningen.
    \end{enumerate}
  \end{example}

  \pause{}

  \begin{block}{Egenskaper}
    \begin{itemize}
      \item Autentiseringen brister, tentanden kan skriva vilket namn och 
        personnumer som helst på inlämnad tentamen.
      \item Ingen anonymitet.
    \end{itemize}
  \end{block}
\end{frame}

\begin{frame}
  \begin{example}[Autentisera uttag]
    \begin{itemize}
      \item Banker lagrade kontonummer på magnetremsan.
      \item PIN-koden skickades till centrala systemet för verifiering.
    \end{itemize}
  \end{example}

  \pause{}

  \begin{alertblock}{\enquote{Förbättring}}
    Kryptera PIN-koden och lagra den på magnetremsan så att uttagsautomaten kan 
    verifiera den när den inte får kontakt med centrala systemet.
  \end{alertblock}

  \pause{}

  \begin{exercise}
    Några problem?
  \end{exercise}
\end{frame}

\begin{frame}
  \begin{alertblock}{Problem}
    \begin{itemize}
      \item Jag kan byta ut kontonumret men inte ändra koden.
      \item Ändra kontonummer och använd min egen kod för att ta ut från annans 
        konto.
    \end{itemize}
  \end{alertblock}
\end{frame}

\subsection{Autentisering}

\begin{frame}
  \begin{itemize}
    \item Lösenord och PIN-koder är fundamentala metoder för autentisiering.

      \pause{}

    \item Exempelvis 90-talets fjärrlås till bilen:
      \begin{itemize}
        \item Nyckeln skickade serienumret.
        \item Bilen kontrollerade allt den mottog och jämförde med sitt 
          serienummer.
      \end{itemize}

      \pause{}

    \item Kunde angripas med en inspelningsattack.
      \begin{itemize}
        \item Spela in allt som sänds (serienummer).
        \item Spela upp serienummer för att låsa upp.
      \end{itemize}

  \end{itemize}
\end{frame}

\begin{frame}
  \begin{itemize}
    \item Ibland kan ett enkelt lösenord vara rätt väg att gå.

    \item Matkuponger på lunchställen:
      \begin{itemize}
        \item Papperslapp med serienummer.
        \item Kan framställas i kopiatorn.
      \end{itemize}

      \pause{}

    \item Hotet är inte värt högre säkerhet.

  \end{itemize}
\end{frame}


%%%%%%%%%%%%%%%%%%%%%%

\begin{frame}
	\small
  \printbibliography{}
\end{frame}

